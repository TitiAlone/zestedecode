\documentclass{beamer}

\usepackage[utf8x]{inputenc}
\usepackage{listings}
\usepackage[ddmmyyyy]{datetime}

\title{Notions de programmation}
\author{Zeste de Savoir}
\usetheme{zestedesavoir}
\begin{document}

\begin{frame}
  \titlepage
\end{frame}

\section{Notions}

\begin{frame}
    \frametitle{Découverte de l'interpréteur}
    Faire des calculs mathématiques simples :
    \begin{itemize}
    \item \(1+1\)
    \item \(6\times 7\)
    \item \(4-2\)
    \item \((6-4)\times 2\)
    \item \(\frac{ 3,6 }{ 2 }\)
    \end{itemize}

    Exercice : calculer le triple de 5,2.
\end{frame}

\begin{frame}
    \frametitle{Déclaration de variables}
    \textbf{Variable} : boite dans laquelle on peut mettre une valeur.
    \begin{itemize}
        \item Taper \texttt{x=3} puis taper \texttt{x}
        \item Taper \texttt{x+1} puis taper \texttt{x}
        \\ La valeur de \texttt{x} a-t-elle changée ?
        \item Taper \texttt{y=4} puis taper \texttt{y=y+2}
        \\ La valeur de \texttt{y} a-t-elle changée ?
    \end{itemize}
\end{frame}

\begin{frame}[fragile]
    \frametitle{Appeler une fonction}
    \textbf{Fonction} : morceau de code que quelqu'un d'autre a déjà écrit pour vous, qu'on appelle avec son nom pour ne pas le ré-écrire.
    \medbreak
    \begin{lstlisting}[language=python]
    print("Bonjour !")
    \end{lstlisting}
\end{frame}

\begin{frame}[fragile]
    \frametitle{Variables et fonctions pour le Snake}
    Gestions des variables dans le Snake :
    \begin{itemize}
        \item Variables locales : portée limitée
        \item Variables de jeu : disponibles partout
    \end{itemize}
    Gestion des fonctions dans le Snake :
    \begin{itemize}
        \item Fonctions principales : \texttt{initialisation} et \texttt{boucle}
        \item Fonctions internes : \textbf{les méthodes}
    \end{itemize}
    \textbf{Un exemple ?} Créons notre première fenêtre.
\end{frame}

\begin{frame}[fragile]
    \frametitle{Conditions}
    \textbf{Condition} : Expression qui exécute un morceau de code seulement si elle est vérifiée.
    \medbreak
    \begin{lstlisting}[language=python]
    age = 23
    if age >= 18:
    ... print("Tu es majeur")
    \end{lstlisting}
    \medbreak
    Liste des tests utiles :
    
    \texttt{>}  : supérieur à\\
    \texttt{<}  : inférieur à\\
    \texttt{==} : égal à\\
    \texttt{!=} : différent de\\
    \medbreak
\end{frame}

\begin{frame}[fragile]
    \frametitle{Conditions (avec else et elif)}
    Si tu as au moins 18 ans alors tu es majeur.
    Sinon, tu es mineur.
    La majorité internationale est quand à elle atteinte à 21 ans :
    \medbreak
    \begin{lstlisting}[language=python]
    age = 15
    if age >= 21:
    ... print("Majorite internationale")
    elif age >= 18:
    ... print("Tu es majeur")
    else:
    ... print("Tu es mineur")
    \end{lstlisting}
\end{frame}

\begin{frame}
    \frametitle{Conditions pour le Snake}
    \textbf{Exercice} : Affecter une taille au serpent et afficher des messages :
    \begin{itemize}
        \item Tu es un tricheur si \texttt{taille < 3}
        \item Début de partie si \texttt{taille = 3}
        \item Serpent long si \texttt{taille > 3}
    \end{itemize}
    \textbf{Exercice (avec la doc.)} : Fermer la fenêtre lorsque l'on clique sur la croix.
\end{frame}

\begin{frame}[fragile]
    \frametitle{Boucles}
    \textbf{Boucle} : Morceau de code répété un certain nombre de fois.
    \medbreak
    \textbf{Application} :
    \begin{lstlisting}[language=python]
    for carreau in jeu.grille():
    ... if jeu.est_un_bord(carreau):
    ... ... print(carreau)
    \end{lstlisting}
    À vous de jouer maintenant !
\end{frame}

\begin{frame}
    \frametitle{Liste des objectifs}
    \begin{itemize}
        \item Création de la fenêtre et gestion de sa fermeture (déjà fait!)
        \item Affichage des éléments : fond, cactus et serpent (déjà entamé)
        \item Gestion des déplacements du serpent
        \item Ajout de la pomme et gestion des collisions (avec la pomme, le bord et le serpent lui-même)
    \end{itemize}
    \textbf{Bonus} pour les plus avancés :
    \begin{itemize}
        \item Empêcher le serpent d'aller en arrière
        \item Ajout d'un compteur de points
        \item Ajout d'un écran pour rejouer
        \item Optimisation du serpent (ajout de coins)
    \end{itemize}
\end{frame}

\end{document}